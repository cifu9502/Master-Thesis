\chapter{Majorana Fermions \label{chap:Majorana}}

The  Majorana Fermions, so called in the name of the Italian physicist Ettore Majorana, were first defined in the attempt to find a real solution of the Dirac equation. The real field that solves this equation describes a fermion which is its own antiparticle. Hence it has no electric charge  nor mass.  Till these days, no fundamental particle with these characteristics has been observed. However, in the last decade, there has been a huge speculation about the possibility of finding Majorana Fermions as a quasi-particles of certain types of topological superconductors.

The topological materials are a novel type of material that experiences phase transitions without any symmetry breaking. As consequence, they cannot be characterized by the Landau theory. Instead, the phase transition is characterized with a topological order . Since topological characters are discrete parameters that describe non-local features of the material, the phases with this characterization present robust characteristics that are difficult to alter. An example of this is the quantum hall effect which robustness is famous and widely used in high precision devices. 

One of the most promising topological materials is the one dimensional majorana wire. This wire is inspired in a famous Kitaev's toy model representing a spinless p-wave superconducting wire. Under certain conditions, the majorana wires experience topological phase transition characterized by the emergence exotic zero-modes localized at edges of the wire. Kitaev associated these modes with majorana quasi-particles  appearing at the boundary of the topological superconducting wire. Moreover, he pointed out that the combined used of robustness from topological materials and Majorana's non-abelian statistics could lead to the creation of fault tolerant quantum gates. This fact opened the doors to the search of majorana fermions in condensed matter physics. 

In this chapter we will present a review of the main topics about majorana fermions. In the first section \ref{sec:KitaevChain}, we will describe the the Kitaev's chain and the emergence of majorana zero modes. Next, we will discuss about the real implementations of majorana chains and the experimental proposals that have been carried on . Finally, we will take a look to the results coupling QDs with majorana chains.


% ------------------------Section: Kitaev *------------------
\section{The Kitaev Chain \label{sec:KitaevChain}}
In this section we discuss the main aspects of Kitaev's toy model. This is a toy model in tight binding that represents a  finite $p$-wave superconducting wire with the following Hamiltonian

\begin{equation}
H = \sum_{i=1}^N \left[ -t(a_i^{\dagger} a_{i+1} + a_{i+1}^{\dagger}a_i) -\mu a_i^{\dagger} a_{i} +  \Delta a_{i}a_{i+1} + \Delta^* a_{i+1}^{\dagger}a_i^{\dagger} \right].  \label{eq:kitaevHam}
\end{equation}

Where $\mu$ is the chemical potential, so that $\mu a_i^{\dagger} a_{i}$ is the energy associated to each step in the chain. $t(a_i^{\dagger} a_{i+1} + a_{i+1}^{\dagger}a_i)$ represents the interaction between neighbouring sites which is determined by the hopping term $t$. The remaining terms describe the superconducting properties of the system as is is established by the BCS theory of superconductivity. $\Delta$ is a complex superconducting parameter with the form  $\Delta = e^{i\theta} \super$. The associated terms represent the Cooper pairs which can be created or annihilated at neighbouring sites of the system.

The form of hamiltonian \prettyref{eq:kitaevHam} favors the possibility of introducing new operators $\gammaA{j}$ and $\gammaB{j}$ such that

\begin{equation}
\gammaA{j} = e^{i\theta /2}a_j+ e^{-i\theta/2 } \ann_j \ \ , \ \ \gammaB{j} = -i(e^{i\theta /2}a_j - e^{-i\theta/2} \ann_j).
\label{eq:majoranaTrans}
\end{equation}
It is simple check that these operators are self-adjoint $(\gammaA{j}^\dagger = \gammaA{j}, \gammaA{j}^\dagger = \gammaB{j})$. This is a required constraint for the Majorana particles. In addition they satisfy the fermionic anti-commutation relations
\begin{equation}
\begin{aligned}
\{\gammaA{i}, \gammaA{j}\} = \{ & \gammaB{i} , \gammaB{j}\} = 2\delta_{ij}  ,\\ 
  \{\gammaA{i}, \gammaB{j} & \} =0.
\end{aligned} 
\label{majoranaRel}
\end{equation} 
This allows us to understand the operators $\gammaA{i} , \gammaB{i}$ as majorana fermions. If we also take the inverse of \prettyref{eq:majoranaTrans} we obtain that each  (Dirac) fermion in Hamiltonian \eqref{eq:kitaevHam} is composed by two majorana fermions such that 
$$a_j = \frac{e^{-i\theta/2}}{2}(\gammaA{j}+ i\gammaB{j})$$
We could even adventure to say that these majorana operators are actually dividing the Dirac fermions into real($\gammaA{}$) and imaginary $(\gammaB{})$ part ,the same way as complex numbers are a composite of two real numbers. 

The new Kitaev Hamiltonian in the Majorana representation looks like 

\begin{equation}
H = \frac{i}{2} \sum_{j=1}^N \left[ -\mu \gammaA{j}\gammaB{j}  + (t- \super) \gammaB{j}\gammaA{j+1} + (t+ \super) \gammaA{j}\gammaB{j+1} \right]+Const,\label{eq:HamMajorana}
\end{equation}

Depending on the values of parameters $\mu, t$ and $\super$ we can identify two regimes represented by the following situations:


%\begin{figure}[t]
%$$\includegraphics[scale=0.5]{KitaevtopPhases.jpg}
%\centering
%\label{top.phases kitaev}
%\caption{{\small \textit{Taken from \cite{bernevig2015topological}. Ilustration of the Kitaev chain for open boundary conditions in the Majorana representation. a)Represents the trivial case where the hopping and the superconducting term approaches to $0$. b) The non-trivial topological phase. The coupling is produced between Majoranas in different Dirac fermions }}}
%\end{figure}

\begin{figure}[hbt]
    \centering
    \includegraphics[scale=0.5]{IMAGES/Majorana/KitaevChain.png}
    \label{fig:top.phases kitaev}
    \caption{Illustration of the Kitaev chain for open boundary conditions in the Majorana representation. a)Represents the trivial case where the hopping and the superconducting term approaches to $0$. b) The non-trivial topological phase. The coupling is produced between Majoranas in different Dirac fermions \protect\Source{By the author} }
\end{figure}


\begin{enumerate}
\item{If $\super = t = 0, \mu <0$} Hamiltonian \eqref{eq:HamMajorana} becomes $\frac{-i\mu}{2} \sum_{j} \gammaA{j}\gammaB{j}$ which represents the coupling of the Majoranas in the same Dirac fermion. (See Figure \ref{fig:top.phases kitaev} (a))

\item{If $\super = t > 0, \mu =0$} the situation is much more interesting. The Hamiltonian \eqref{eq:HamMajorana} takes the form $H = 2ti\sum_{j} \gammaA{j}\gammaB{j+1}$. This implies that the coupling is performed between  Majoranas of different Dirac fermions leaving the edge Majorana operators ($\gammaA{1}$ and $\gammaB{N}$) uncoupled (See Figure \ref{fig:top.phases kitaev}b)). Note that these uncoupled majorana fermions can be at any state without any  repercussion in the energy of the system. This explains the emergence of a  ground state localized at edges of the chain. 
\end{enumerate}

These two situations are representatives of two different phases. The trivial phase occurs for $\frac{\mu}{2t}>1$ and the non-trivial phase appears when $\frac{\mu}{2t}<1$ (See figure \ref{fig:KitaevSpec}). The mean characteristic of the non-trivial phase is the creation of an stable zero-mode. This zero-mode is generated by the  uncoupled majorana fermions at the edges of the Kitaev chain.  \\



\begin{figure}[t]
    \centering
    \includegraphics[scale=0.5]{IMAGES/Majorana/Spectrum.png}
    \label{fig:KitaevSpec}
    \caption{Spectrum of Hamiltonian \eqref{eq:HamMajorana} with $30$ sites and $t=\super$ s. Method: Numerical diagonalization. \protect \Source{By the author} }
\end{figure}

\subsection{Topological phase transition}

The two regimes described previously  can be characterized with a topological parameter.  One of the methods for this is following the idea used by \citeauthor{alicea_new_2012}\cite{alicea_new_2012}. The first part is to suppose that we have an infinite chain $(N=\infty)$ in Hamiltonian \eqref{eq:HamMajorana}. The new system is translation invariant, hence we can make a transformation to the momentum space. Then we may rewrite Hamiltonian \eqref{eq:HamMajorana}  as

\begin{equation}
    H = 
    \sum_{k \in BZ} 
    \begin{pmatrix} 
      b_k'  & c_{k}'\\  
    \end{pmatrix}
    H_k 
    \begin{pmatrix} 
      b_{-k}'     \\ 
      c_{-k}' 
    \end{pmatrix}.
    \label{PBCHam2}
\end{equation}

with the Bloch Hamiltonian 

\begin{equation}
H_k = \begin{pmatrix} 
      0    &  \frac{-i \mu}{2} + it \cos k + \super  \sin k  \\ 
       \frac{i \mu}{2} - it \cos k + \super \sin k  &  0 
    \end{pmatrix}
    = (\super \sin k) \sigma_x + (\frac{\mu}{2}- t \cos k) \sigma_y.
\label{sigma}
\end{equation}




\noindent where $\sigma_x$ , $\sigma_y$ are the corresponding Pauli matrices. The Brilloin zone ($BZ$) is the periodic space  $[-\pi , \pi]$ which can be mapped to the unitary circle.   Equation \eqref{sigma} determines  the coordinates of the Bloch Hamiltonian in the base $\{\sigma_x, \sigma_y\}$. We can map these coordinates to the unitary circle by taking the norm of this vector giving
\begin{equation}
     \hat{H}_k= \frac{1}{\sqrt{\super^2 \sin^2 k + (\frac{\mu}{2}- t \cos k)^2}}
     \begin{pmatrix} 
      \super \sin k    \\ 
      \frac{\mu}{2}- t \cos k 
    \end{pmatrix}. 
\end{equation}

Note that $\super^2 \sin^2 k + (\frac{\mu}{2}- t \cos k)^2 \neq 0$ for all the values of $k$ as long as $\frac{\mu}{2t} \neq 1$ . When $\frac{\mu}{2t} = 1$ the $H_{k=0}=0$, so it cannot be normalized. \textbf{This is the same point were the phase transition occurs!}. At any other value of $\frac{\mu}{2t}$ it is possible to normalize $H_{k}$ for all values of $k\in BZ$. The result of mapping $\hat{H}_k$ for all $k$ is a path around the unitary circle. This path can take two forms.






Since we have only the matrices $\sigma_x$ and $\sigma_y$, we can think of $H_k$ as an element of the unitary circle $\mathbb{S}^1$, described by the normalization of the vector  $(\super \sin k, \frac{\mu}{2}- t \cos k)$. However, not all the elements in $\mathbb{S}^1$ will be a representation of $H_k$ for some $k$. To see this, let us take the path in $\mathbb{S}^1$ given by the elements of $H_k$ for $k \in [- \pi , \pi]$.

First note that for $k = 0$, $H_0$ corresponds to the normalization of $(0,\frac{\mu}{2}- t)$ which is simply $(0,1)$ or $(0,-1)$ depending on the case when $\frac{\mu}{2}- t$  is greater or less than $0$, respectively. This corresponds to saying that $\lambda$ is greater or less than $1$ in each case.  Meanwhile, when $k = \pi = -\pi $ the corresponding vector in $\mathbb{S}^1$ can only be $(0,1)$. \\

Again, our problem is divided in two phases. In the trivial phase $(\lambda >1)$ we will obtain a path in $\mathbb{S}^1$ that starts in $(0,1)$  at $k = -\pi $ and passes again through $(0,1)$ at $k = 0$. It is simple to see that the point $(1,0)$ is not in this path since it would imply that $\frac{\mu}{2}- t \cos k = 0$ for some $k$ which is impossible in the trivial phase. See figure \ref{top} (a).

In the non-trivial phase  $(\lambda < 1)$ the situation is completely different. Since we have that $H_0$ corresponds to the point $(0,-1)$ in $\mathbb{S}^1$ and $H_{\pi} = H_{- \pi}$ corresponds to $(0,1)$. It follows that in the non-trivial case the path given by $k \in [- \pi , \pi]$ corresponds to a closed loop around the circle that starts in the north pole $(0,1)$, passes through the south pole $(0,-1)$ at $k = 0$ and then comes back to the north.  See figure \ref{top} (b).


\begin{figure}[t]
\includegraphics[width = 2.9in]{IMAGES/Majorana}}
\label{top}
\caption{{\small \textit{The following represents the path of $H_k$ for the interval $[ -\pi, \pi ]$. $a)$ Corresponds to the trivial phase. The resulting path can be homotopically deformed to a point. $b)$ The non-trivial phase corresponds to a non-contractible loop around $\mathbb{S}^1$.} }}
\end{figure}

Figure \ref{top} represents the first topological description that we have for the phase transition in the Kitaev chain. The topological order that characterizes this chain will be a $\mathbb{Z}_2$ invariant:
\begin{itemize}
\item $0$ for the trivial phase since the final loop is contractible. 
\item $1$ for the non-trivial phase since it corresponds to a non-contractible cycle around $\mathbb{S}^1$.
\end{itemize}

The mean idea behind this relies in the adiabatic theorem.  In simple words, the adiabatic theorem says that a slow evolution of a gaped Hamiltonian will produce a smooth evolution of its ordered eigenstates. i.g The order of the eigenstates remains unchanged. \\

The keyword in the previous definition is "gaped". As we can observe in Figure \ref{fig:KitaevSpec} the phase transition occurs at $\frac{\mu}{2t}=1$. This is when the system transitions between a gapless and gaped Hamiltonians.
The connection with topology comes from the fact that adiabatic evolutions can be understood as smooth deformations of the Hamiltonian. However since gapless Hamiltonians imply phase transitions, the theory defines the gapless points as holes (or forbidden points) in the phase space. Then characterizing the phase transitions in the Kitaev chain is mainly a topological problem where gaped Hamiltonians are holes in the topological space. In addition, the topological orders characterizing this transition will be Chern or Winding numbers. \\


Though this connection between physics and topology is quite interesting, I will stop here because it is taking us out of our real discussion which is majorana fermions.  You can find more information about this in ( \Jesus{add references}). However for this case the  \\




% -----------------------Section: Modern and Experimental-------------
\section{Real implementations of majorana chains}


% -----------------------Leaking majorana modes in quantum dots-------------
\section{Leaking of Majorana modes in QDs}


