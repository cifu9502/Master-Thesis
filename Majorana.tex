\chapter{Motivation}
\section{Majorana Fermions}

The  Majorana Fermions, so called in the name of the Italian physicist Ettore Majorana, where first defined in the attempt to find a real solution of the Dirac equation. The real field that solves this equation $\Psi_M$ , describes a fermion which is its own antiparticle. Hence it has no electric charge nor mass.  Till these days, no fundamental particle with these characteristics has been observed. However, in the last few years, there has been a huge speculation about the possibility of finding Majorana Fermions as a quasiparticle inside certain condensed matter systems. 

One of the most famous examples of these systems is the Kitaev chain which is the main objective of the this subsection. 


\section{The Kitaev Chain}
The Kitaev chain is a toy model in tight binding that represents a  finite $p$-wave superconducting wire. The main Hamiltonian is given by 
\begin{equation}
H = \sum_{i} \left[ -t(a_i^{\dagger} a_{i+1} + a_{i+1}^{\dagger}a_i) -\mu a_i^{\dagger} a_{i} +  \Delta a_{i}a_{i+1} + \Delta^* a_{i+1}^{\dagger}a_i^{\dagger} \right].  \label{eq:kitaevHam}
\end{equation}

Where $\mu$ is the chemical potential, so that $\mu a_i^{\dagger} a_{i}$ is the energy associated to each free state. $t(a_i^{\dagger} a_{i+1} + a_{i+1}^{\dagger}a_i)$ represents the interaction between neighbouring sites which is determined by the hopping term $t$. The remaining terms describe the superconducting properties of the system as is is established by the BCS theory of superconductivity. $\Delta$ is a complex superconducting parameter with the form  $\Delta = e^{i\theta} \super$. The associated terms represent the Cooper pairs which can be created or annihilated at neighbouring sites of the system.

The form of hamiltonian \prettyref{eq:kitaevHam} favors the possibility of introducing new operators $\gammaA{j}$ and $\gammaB{j}$ such that

\begin{equation}
\gammaA{j} = e^{i\theta /2}a_j+ e^{-i\theta/2 } \ann_j \ \ , \ \ \gammaB{j} = -i(e^{i\theta /2}a_j - e^{-i\theta/2} \ann_j).
\label{eq:majoranaTrans}
\end{equation}
It is simple check that these operators are self-adjoint $(\gammaA{j}^\dagger = \gammaA{j}, \gammaA{j}^\dagger = \gammaB{j})$. This is a required constraint for the Majorana particles. In addition they satisfy the fermionic anti-commutation relations
\begin{equation}
\begin{aligned}
\{\gammaA{i}, \gammaA{j}\} = \{ & \gammaB{i} , \gammaB{j}\} = 2\delta_{ij}  ,\\ 
  \{\gammaA{i}, \gammaB{j} & \} =0.
\end{aligned} 
\label{majoranaRel}
\end{equation} 
This allows us to understand the operators $\gammaA{i} , \gammaB{i}$ as majorana fermions. If we also take the inverse of \prettyref{eq:majoranaTrans} we obtain that each  (Dirac) fermion in Hamiltonian \eqref{eq:kitaevHam} is composed by two majorana fermions such that 
$$a_j = \frac{e^{-i\theta/2}}{2}(\gammaA{j}+ i\gammaB{j})$$
We could even adventure to say that these majorana operators are actually dividing the Dirac fermions into real($\gammaA{}$) and imaginary $(\gammaB{})$ part ,the same way as complex numbers are a composite of two real numbers. 

The new Kitaev Hamiltonian in the Majorana representation looks like 

\begin{equation}
H = \frac{i}{2} \sum_{j} \left[ -\mu \gammaA{j}\gammaB{j}  + (t- \super) \gammaB{j}\gammaA{j+1} + (t+ \super) \gammaA{j}\gammaB{j+1} \right]+Const,\label{eq:HamMajorana}
\end{equation}

Depending on the values of parameters $\mu, t$ and $\super$ we can identify two regimes represented by the following situations:


%\begin{figure}[t]
%$$\includegraphics[scale=0.5]{KitaevtopPhases.jpg}
%\centering
%\label{top.phases kitaev}
%\caption{{\small \textit{Taken from \cite{bernevig2015topological}. Ilustration of the Kitaev chain for open boundary conditions in the Majorana representation. a)Represents the trivial case where the hopping and the superconducting term approaches to $0$. b) The non-trivial topological phase. The coupling is produced between Majoranas in different Dirac fermions }}}
%\end{figure}


\begin{enumerate}
\item{If $\super = t = 0, \mu <0$} Hamiltonian \eqref{eq:HamMajorana} becomes $\frac{-i\mu}{2} \sum_{j} \gammaA{j}\gammaB{j}$ which represents the coupling of the Majoranas in the same Dirac fermion. (See figure \ref{top.phases kitaev} (a))

\item{If $\super = t > 0, \mu =0$} the situation is much more interesting. The Hamiltonian \eqref{HamMajorana} takes the form $H = 2ti\sum_{j} \gammaA{j}\gammaB{j+1}$. This implies that the coupling is performed between  Majoranas of different Dirac fermions leaving the edge Majorana operators ($\gammaA{1}$ and $\gammaA{2}$) uncoupled . This produces a new degeneracy in the ground state due to the emergence of a state produced by the uncoupled Majorana operators. The new state is localized at the edges of the chain.(See figure \ref{top.phases kitaev} (b)) 
\end{enumerate}
