\chapter{Double QD coupled to a Majorana Bound State}

%\begin{figure}[bh]
% \includegraphics[scale=0.4]{IMAGES/2Dot-chain.eps}\caption{\label{Fig_2QD-Majorana} Two Qds ($a$ \& $b$) coupled to a TS sustaining
% a Majorana Bound State(MBS) at the edge. To the other side they are
% coupled to a methallic reservoir where conductivity is measured. }


% \end{figure}


We now intend to study the transport properties through two QDs that
are coupled to a topological superconducting (TS) chain sustaining
a Majorana Bound State (MBS) as it is observe in \ref{Fig_2QD-Majorana}.
In \prettyref{sec:The-Numerical-Renormaliztion} we saw how the NRG
code can be applied to study the physics of transport through a QD.
In our case, we can set-up a similar Anderson-model to the one used
in \ref{eq:hamB0} taking


The dimensionality of this system is $4\times4\times2=32.$ Again
we can write this hamiltonian by blocks using the preserved symmetries.
This time we can observe that the number of $\uparrow$-particles
$\left(N_{\uparrow}\right)$ is preserved in \ref{eqFinalMJ-2QDs},
but it is not for $\downarrow$-particles due to the terms $\left(d_{i\downarrow}^{\dagger}f_{\downarrow}^{\dagger},\ f_{\downarrow}d_{i\downarrow}\right)$.
However, the parity of $\downarrow$-particle $\left(P_{\downarrow}\right)$
is always preserved since $\left(d_{i\downarrow}^{\dagger}f_{\downarrow}^{\dagger},\ f_{\downarrow}d_{i\downarrow}\right)$
create or annihilate $2$-particles in the system. 

The final computations for $H_{TS-2QDs}$ in terms of the $N_{\uparrow},P_{\downarrow}$-symmetry
can be found in \prettyref{chap:Double-Dot-Majorana-Hamiltonian.}.
Setting $H_{-1}=H_{TS-2QDs}$ we can use the NRG algorithm \prettyref{eq:NRG-Iteration Hamiltonians}
to iteratively diagonalize this hamiltonian by 

\[
H_{N+1}=\Lambda^{\frac{1}{2}}\left[H_{N}+\frac{1}{2}\left(1+\Lambda^{-1}\right)\xi_{N}\left(f_{N\sigma}^{\dagger}f_{N+1,\sigma}+f_{N+1\sigma}^{\dagger}f_{N\sigma}\right)\right]\ \mbox{for \ensuremath{N\geq0}}.
\]


At $N=-1$ the equation above won't work since there are two QDs coupled
with the leads. The answer to this problem is simply to define constants
$\xi_{0i}$ that characterizes the coupling between dot $i$ and the
first opperator $f_{0}^{\dagger}.$ Thus we obtain 

\[
H_{0}=\Lambda^{\frac{1}{2}}\left[H_{-1}+\frac{1}{2}\left(1+\Lambda^{-1}\right)\sum_{i}\xi_{i0}\left(i_{i\sigma}^{\dagger}f_{0,\sigma}+f_{0\sigma}^{\dagger}d_{i\sigma}\right)\right]\ \mbox{for \ensuremath{N\geq0}}.
\]


This idea completes the NRG algorithm for the $2$QD-TS model coupled
to metallic lead. An NRG extension of the code developed by my thesis
advisor has been implemented \footnote{The code can be found in \url{https://github.com/cifu9502/nrgcode}}
and it is now in the error-correction stage. We hope for a rapid correction
of these errors to start running the program. 



\subsection{Atomic limit For Separate Dots ($\tdots=0$)}
\label{sec:AtomicLimit}


In the atomic limit $(\Gamma = 0)$ the lead interaction is neglected. Hence the Hamiltonian \eqref{eq:Ham} reduces to 

    \begin{equation}
        H=H_{DQD}+H_{M-DQDs}.
    \end{equation}
    
The dimension of this Hamiltonian is $2^2\times2^2\times 2 =32$ ($2^2$ per QD $\up$, $\dw$ and $2$ due to the majorana spin-$\dw$). 

A simple analytical solution can be given for the case where  $\ed{i}=\frac{-U_i}{2}=\frac{-U}{2}$ and $\tdots = 0$. With

    \begin{eqnarray}
        H=  & \frac{U}{2}\sum_i(\sum_{\sigma} \hat{n}_{i\sigma}-1)^{2} +  \sum_{i} t_i \left(d_{i\downarrow}^{\dagger}\gamma_{1}+\gamma_{1}d_{i\downarrow}\right).
        \label{eq:AtomicHam}
    \end{eqnarray}

    % \begin{eqnarray}
    %     H=  & \frac{U}{2}\sum_i(\sum_{\sigma} \hat{n}_{i\sigma}-1)^{2} + t \sum_{i} \left(d_{i\downarrow}^{\dagger}\gamma_{1}+\gamma_{1}d_{i\downarrow}\right).
    %     \label{eq:AtomicHam}
    % \end{eqnarray}

Hamiltonian \eqref{eq:AtomicHam} can be written in blocks labeled by the  two conserved quantum numbers $\hat{N}_\up $ and $\hat{P}_\dw = \pm$ . For example the block  for $0$ spin-$\up$  and odd spin-$\dw$ particles  $\left( \hat{N}_\up =0, \hat{P}_\dw = - \right)$ can be written in terms of the base 
    \begin{equation}
        \lbrace \ket{\dw,\dw,\dw} , \ket{0,0,\dw} , \ket{0,\dw,0}, \ket{\dw,0,0}     \rbrace;
    \end{equation}
where the states are labeled by  $\ket{QD1,QD2,MZM}$. Hence the representation of block     $\left( \hat{N}_\up =0, \hat{P}_\dw = - \right)$ is 
    \begin{equation}
    \begin{array}{c}
        \vert\downarrow,\downarrow,\downarrow\rangle\rightarrow\\
        \vert0,0,\downarrow\rangle\rightarrow\\
        \vert0,\downarrow,0\rangle\rightarrow\\
        \vert\downarrow,0,0\rangle\rightarrow
        \end{array}\left[\begin{array}{cccc}
        0 & 0 & -t_1 & t_2\\
        0 & U & t_2 & t_1\\
        -t_1 & t_2 & \frac{U}{2} & 0\\
        t_2 & t_1 & 0 & \frac{U}{2}
    \end{array}\right].
    \end{equation}

The four eigen-energies of this block of the Hamiltonian can be written as \LUIS{Let's call these energy states 
%
    \begin{eqnarray}
        \varepsilon^{(1)}_{\pm} = & U/4 \pm  \sqrt{\frac{U^2}{16} + t_1^2+t_2^2} \\ \nonumber 
        \varepsilon^{(2)}_{\pm} = &  \frac{3U}{4} \pm  \sqrt{\frac{U^2}{16}+ t_1^2+t_2^2} \; .
        \label{eq:atomiclimit1}
    \end{eqnarray}
    % \begin{equation}
    %     U/4 \pm  \sqrt{\frac{U^2}{16} + 2t^2} \ , \ 3U/4 \pm  \sqrt{\frac{U^2}{16}+ 2t^2}.
    % \end{equation}
For $t_{1,2} << \frac{U}{4}$, the eigenvalues can be approximated by the Taylor series giving
    \begin{eqnarray}
        \varepsilon^{(1)}_{\pm} \approx & U/4 \pm  U/4 \left(1 + 8\frac{t_1^2+t_2^2}{U^2} \right) \\ \nonumber
        \varepsilon^{(2)}_{\pm} \approx &  \frac{3U}{4} \pm  U/4 \left(1 + 8\frac{t_1^2+t_2^2}{U^2} \right) \; .
        \label{eq:atomiclimit2}
    \end{eqnarray} 
    % \begin{equation}
    %     U/4 \pm  U/4 \left(1 + \frac{16t^2}{U^2} \right)+  \ , \ 3U/4 \pm  U/4 \left(1 + \frac{16t^2}{U^2} \right).
    % \end{equation}

Comparing this results with the $t=0$ eigen-values $(0,\frac{U}{2},U)$, we observe that the displacement of the energy levels  generated by the majorana couplings $t_1$ and $t_2$ is 

    \begin{equation}
        \delta_{t_1,t_2} = \frac{2t_1^2+2t_2^2}{U}.
        \label{eq:displacement}
    \end{equation}

This implies that the energy scale where the majorana effects will be observed will scale as the square root of the energy couplings. A similar result will be obtained for the other quantum numbers. 

